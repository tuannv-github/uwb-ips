\providecommand{\main}{..}
\documentclass[\main/main.tex]{subfiles}

\begin{document}
\graphicspath{{img/}{01_intro/img/}}

\chapter{Introduction}

\section{Problem statement}
Localization has always been in humans problem throughout history. From the old days when people used to follow the ancient guiding-star navigation until now, a lot of technology has been introduced and finally, Global Positioning System (GPS) has practically solved the problem of outdoor localization. As GPS makes use of satellites, which are located thousand miles away from the ground, signals from satellites are obstructed on way from satellites to devices, result in weak signals. Different barriers such as trees and buildings reflecting weak signals also cause multi-path interference. Moreover, the building materials may make it more difficult to perform indoor tracking through GPS. As a result, to provide a GPS-like system for the indoor environment, a lot of studies have been done to develop a full-featured effective and accurate Indoor Positioning System (IPS).
\newline\newline
A navigation system which is made of network devices to locate objects or people with capability to localize the position of a wireless capable device within a particular space inside indoor environment is referred as IPS \cite{survey_on_indoor_wireless_positioning_techniques}. After a great success in adopting with GPS, developing IPS has become a popular research area due to its increasing demand. People want to use indoor positioning system for various purposes such as security, finding location of materials and emergency. 
\newline\newline
Many techniques can be used for localization \cite{a_survey_on_localization_for_mobile_wireless_sensor_networks}. Traditionally, different infrastructure such as Wi-Fi, Bluetooth are re-purpose to estimate location for indoor environments. The common approach to estimating distance or location with Wi-Fi and Bluetooth signals is to measure signal strength. The problem with such an approach, of course, is that signal strength is a poor indicator of distance.
\newline\newline
Ultra-wideband (UWB) is a radio technology which makes extremely precise indoor positioning possible. Goods, vehicles and machines can be located with an accuracy of 10-30 centimeters compared with Bluetooth (1-3 meters) or Wi-Fi (5-15 meters). The position determination is carried out by means of a run-time method (Time of Flight, ToF). For this, at least three so-called anchors are required. The radiated light between them and the asset is measured by tags attached to the asset. The position can be calculated from such measurements.
\newline\newline
The DW1000 is a fully integrated single chip Ultra Wideband (UWB) low-power low-cost transceiver IC compliant to IEEE802.15.4-2011. It can be used in 2-way ranging or TDOA location systems to locate assets to a precision of 10 cm. It also supports data transfer at rates up to 6.8 Mbps \cite{decawave:dw1000_datasheet}.
\newline\newline
The DWM1001C module combines the DW1000, a Nordic Semiconductor nRF52832 MCU, and a 3-axis accelerometer. Using this module accelerates design cycle, reduces development costs and shortens time to market.

\section{Thesis objective}
The main objective of this thesis is to design and implement an indoor localization system using the UWB-based DWM1001C module. In addition, a system based on a Bluetooth mesh network is also built to manage and control the localization system. The mesh network also provides some IoT services for remote lighting control and self identification.

\begin{figure}[H]
    \begin{center}
        \includegraphics[width=0.9\textwidth]{fonctionnement-technologie-mesh-wirepas.jpg}
    \end{center}
    \caption{Indoor location}
    \label{fig:indoor_location}
\end{figure}

\bib
\end{document}